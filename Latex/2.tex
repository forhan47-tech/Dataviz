% ============================
% Basic LaTeX Document Body
% ============================

\documentclass[12pt,a4paper]{article}

% ---------- Preamble ----------
\usepackage[utf8]{inputenc}
\usepackage[T1]{fontenc}
\usepackage[a4paper,margin=1in]{geometry}
\usepackage{amsmath}
\usepackage{graphicx}
\usepackage{hyperref}


% ---------- Begin Document ----------
\begin{document}
\textbf{প্রশ্ন ১ :}\\
২৫ kg ভরের এবং ১৫ m দৈর্ঘ্যের একটি বস্তু স্থির অবস্থা থেকে ০.৫ c বেগে গতি করছে।\\
(গ) গতিশীল অবস্থায় বস্তুর দৈর্ঘ্য কত?\\
(ঘ) নিউটনীয় বলবিদ্যায় প্রাপ্ত গতিশক্তি এবং আপেক্ষিক তত্ত্বে প্রাপ্ত গতিশক্তি সমান নয় — গাণিতিকভাবে যাচাই কর।\\[1em]

\textbf{প্রশ্ন ২ :}\\
একটি পরীক্ষায় ০.৪ Å তরঙ্গদৈর্ঘ্যের একটি ফোটন একটি স্থির ইলেকট্রনের ওপর আপতিত হয়ে ৫৫° কোণে বিচ্ছুরিত হয়।\\
(গবেষণাগারে তালিকা থেকে জানা যায়:\\
ইলেকট্রনের ভর = ৯.১ × ১০⁻³¹ kg,\\
আলোর বেগ = ৩ × ১০⁸ m/s,\\
এবং প্লাঙ্ক ধ্রুবক = ৬.৬৩ × ১০⁻³⁴ J·s)\\
(গ) আপতিত ফোটনের শক্তি নির্ণয় কর।\\
(ঘ) বিকিরণের আগে ও পরে ফোটনের শক্তি তুলনা কর।\\[1em]

\textbf{প্রশ্ন ৩ :}\\
একটি ফুটবল খেলার মাঠের দৈর্ঘ্য ৪০০ m এবং প্রস্থ ২০০ m।\\
একজন নভোচারী ০.৮৬ c বেগে ১০০০ kg ভরযুক্ত একটি নভোযানে চড়ে মাঠটির দৈর্ঘ্য বরাবর অতিক্রম করছেন।\\
(i) নভোযানটির আপাত ভর নির্ণয় কর।\\
(ii) নভোচারীর দৃষ্টিতে মাঠের দৈর্ঘ্য মাপের প্রকৃত মাপের সমান নয় — গাণিতিক বিশ্লেষণের মাধ্যমে যাচাই কর।\\

\end{document}

