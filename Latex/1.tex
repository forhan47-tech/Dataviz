% Compile with LuaLaTeX
\documentclass[a4paper,12pt]{article}

%==============================
% Packages
%==============================
\usepackage{amsmath}
\usepackage{amssymb}
\usepackage{siunitx}          % For clean units
\usepackage{polyglossia}
\setdefaultlanguage[numerals=western]{bengali}
\setotherlanguage{english}

%==============================
% Font settings
%==============================
% Use the family name exactly as seen in fc-list
\newfontfamily\bengalifont[
  Script=Bengali,
  Ligatures=TeX
]{NotoSansBengali}

% Optional fallback if you prefer UI variant
% \newfontfamily\bengalifont[
%   Script=Bengali,
%   Ligatures=TeX
% ]{NotoSansBengaliUI}

\setmainfont{Latin Modern Roman} % Latin text

% Convenience macro
\newcommand{\bn}[1]{{\bengalifont #1}}

%==============================
% Document
%==============================
\begin{document}

\author{}
\date{}
\maketitle

\section*{\bn{প্রশ্ন ১}}
\bn{২৫ kg ভরের এবং ১৫ m দৈর্ঘ্যের একটি বস্তু স্থির অবস্থা থেকে ০.৫ c বেগে গতি করছে।}
\begin{enumerate}
  \item[\bn{(গ)}] \bn{গতিশীল অবস্থায় বস্তুর দৈর্ঘ্য কত?}
  \item[\bn{(ঘ)}] \bn{নিউটনীয় বলবিদ্যায় প্রাপ্ত গতিশক্তি এবং আপেক্ষিক তত্ত্বে প্রাপ্ত গতিশক্তি সমান নয় — গাণিতিকভাবে যাচাই কর।}
\end{enumerate}

\section*{\bn{প্রশ্ন ২}}
\bn{একটি পরীক্ষায় ০.৪ Å তরঙ্গদৈর্ঘ্যের একটি ফোটন একটি স্থির ইলেকট্রনের ওপর আপতিত হয়ে ৫৫° কোণে বিচ্ছুরিত হয়।}
\begin{itemize}
  \item[\bn{(গ)}] \bn{আপতিত ফোটনের শক্তি নির্ণয় কর।}
  \item[\bn{(ঘ)}] \bn{বিকিরণের আগে ও পরে ফোটনের শক্তি তুলনা কর।}
\end{itemize}

\bn{(গবেষণাগারে তালিকা থেকে জানা যায়: ইলেকট্রনের ভর = $9.1 \times 10^{-31}$ kg, আলোর বেগ = $3 \times 10^8$ m/s, এবং প্লাঙ্ক ধ্রুবক = $6.63 \times 10^{-34}$ J$\cdot$s)}

\section*{\bn{প্রশ্ন ৩}}
\bn{একটি ফুটবল খেলার মাঠের দৈর্ঘ্য ৪০০ m এবং প্রস্থ ২০০ m। একজন নভোচারী ০.৮৬ c বেগে ১০০০ kg ভরযুক্ত একটি নভোযানে চড়ে মাঠটির দৈর্ঘ্য বরাবর অতিক্রম করছেন।}
\begin{enumerate}
  \item[\bn{(i)}] \bn{নভোযানটির আপাত ভর নির্ণয় কর।}
  \item[\bn{(ii)}] \bn{নভোচারীর দৃষ্টিতে মাঠের দৈর্ঘ্য মাপের প্রকৃত মাপের সমান নয় — গাণিতিক বিশ্লেষণের মাধ্যমে যাচাই কর।}
\end{enumerate}

\end{document}
